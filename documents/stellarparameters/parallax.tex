%% This file is part of the Gaia project.
%% Copyright 2016 the authors.

\documentclass[12pt, preprint]{aastex}

\newcommand{\project}[1]{\textsl{#1}}
\newcommand{\gaia}{\project{Gaia}}

\newcommand{\logg}{\log g}
\newcommand{\Teff}{T_{\mathrm{eff}}}

\begin{document}

\title{What can you learn about a star from its parallax?}
\author{DWH, HWR, others}

\begin{abstract}
Perhaps the most physically informative measurement to take of
an unresolved star is a high resolution spectrum, which can be used
to constrain the local stellar photospheric parameters (effective
temperature and surface gravity to first order), using a stellar
photosphere model.
These infererences can be helped by, but do not in any way require
the use of, a stellar evolution model or tracks.
Spectroscopic surveys often choose to model stellar
spectra using photosphere models only, with no input of evolution
models.
The \gaia\ Mission is about to release precise parallaxes for hundreds
of thousands of stars!
When combined with photometry or spectrophotometry, these parallaxes
deliver information about bolometric luminosity and surface gravity;
they hold great promise for improving inferences about stars.
Here we make the general properties of these inferences explicit with
a graphical model and discussion.
Any use of a parallax
to learn about stellar parameters necessarily requires the use not
just of a photosphere model, but also a stellar-evolution
model.
Once an evolution model is used in conjunction with the parallax
inference, it is inconsistent (and wrong) not to also include the
evolution model in the spectrum inference (that is, require the
effective temperature and surface gravity to lie on evolutionary
tracks or isochrones).
That is, the presence of the \gaia\ data will force changes to how
we analyze spectral data.
We give some unsolicited advice, and perform some demonstrations with toy data.
\end{abstract}

\keywords{
  Hello
  ---
  World
}

Hello World!

\clearpage
\begin{figure}
\includegraphics{./parameters01.pdf}
\caption{Whaty what.\label{good}}
\end{figure}

\clearpage
\begin{figure}
\includegraphics{./parameters02.pdf}
\caption{Whaty what.\label{complex}}
\end{figure}

\clearpage
\begin{figure}
\includegraphics{./parameters00.pdf}
\caption{Whaty what.\label{bad}}
\end{figure}

\end{document}
