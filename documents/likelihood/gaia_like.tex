\documentclass[12pt]{article}
\newcommand{\Gaia}{\textsl{Gaia}}
\begin{document}

\noindent
\textbf{A likelihood function for the \Gaia\ Data}
\bigskip

\noindent
\textbf{David W. Hogg} and others

\paragraph{Abstract:}
When we perform probabilistic inferences with the \Gaia\ Mission data,
we technically require
a \emph{likelihood function}, or a probability of the (raw-ish) data as a function
of stellar (astrometric and photometric) properties.
Unfortunately, we aren't (at present) given access to the \Gaia
data directly;
we are only given a catalog of derived astrometric properties for the stars.
How do we perform probabilistic inferences in this context?
The answer---implicit in many publications---is that we should look at the
\Gaia\ Catalog as containing the parameters of a likelihood function, or
a probability of the \Gaia\ data, conditioned on stellar properties,
evaluated at the location of the data.
Concretely, our recommendation is to assume
(for, say, the parallax) that the \Gaia\ Catalog-reported
value and uncertainty are the mean and root-variance of a Gaussian
function that can stand in for the true likelihood function.
This is the implicit assumption in most \Gaia\ literature to date;
our only goal here is to make the assumption explicit.
Certain technical choices by the \Gaia\ team could invalidate
this assumption; it is important to downstream users of any catalog products
that they can get from those products likelihood information about the fundamental
data.

\paragraph{Introduction:}
The \Gaia\ data are being used in many probabilistic inferences (cite lots of stuff).

These inferences implicitly or explicitly construct a locally Gaussian
likelihood function for each object.

There is work on probabilistic catalogs (cite lots of stuff). This work
will be useless if it can't be used to construct a LF in the end.

\paragraph{A likelihood function for \Gaia}
Can think of this as a LF given the catalog data. Or of the raw data.
These will be identical if the catalog contains a set of sufficient statistics
for the raw data!
Attitude taken here is that we care about the raw data likelihood.

There are nuisance parameters. We are going to assume that the Gaia team either
optimizes or marginalizes over these. The difference between these options won't
be big.

Give the one-d formula. Cite various.

Give the K-d formula. Cite various.

\paragraph{Discussion:}
Why did you write this note? I am not sure. It is just a restate of the
obvious.

What have we assumed? That the \Gaia\ DPAC did LF optmimization. And the
noise is close to Gaussian, and the signal-to-noise is high.

What about star--star covariances? These are being ignored too, but they
will be small in the long run (cite Holl?).

What about non-Gaussianity? There is nothing we can do about that.

Why are there negative parallaxes? There will be, and you can think about
it two ways: Linear fitting will do this at low s/n in general. And the
LF approximation might require it!
They are required if the team is delivering likelihood information.

What are we doing with our inferences? If we are producing true posterior
information (as with cite this and cite that), then our inferences may not
be re-usable by anyone. This is a danger for all of us. And it isn't enough
to report our priors, because: Reproducibility, support, etc.

\end{document}
