\documentclass[12pt]{article}
\begin{document}

\noindent
\textbf{A likelihood function for the \textsl{Gaia} Data}

\noindent
\textbf{David W. Hogg} and others

\paragraph{Abstract:}
When we perform probabilistic inferences with the \textsl{Gaia} Mission data,
we technically require
a \emph{likelihood function}, or a probability of the (raw) data as a function
of stellar (astrometric and photometric) properties.
Unfortunately, we aren't (at present) given access to the raw \textsl{Gaia}
data; we are only given a catalog of derived astrometric properties for the stars.
How are we to perform probabilistic inferences in this context?
The answer---implicit in many publications---is that we should look at the
\textsl{Gaia} Catalog as containing not the \emph{values} of the stellar properties,
nor estimates thereof, but instead as containing parameters of a likelihood function, or
a probability of the \textsl{Gaia} data, conditioned on stellar parameters,
evaluated at the location of the data.
Concretely, our recommendation is to assume
(for, say, the parallax) that the \textsl{Gaia} Catalog-reported
value is the mean and the uncertainty is the root-variance of a Gaussian
function that can stand in for the true likelihood function.
This is the implicit assumption in most \textsl{Gaia} literature to date;
our only goal here is to make that assumption explicit.

\end{document}
