% This file is part of the Gaia project.
% Copyright 2012 David W. Hogg (NYU).

\documentclass[12pt]{article}

\newcommand{\project}[1]{\textsl{#1}}
\newcommand{\gaia}{\project{Gaia}}

\begin{document}\sloppy\sloppypar\raggedbottom

\section*{Controlling the complexity of the \gaia \\
          (or any spacecraft) attitude model}

\noindent
David W. Hogg (NYU) \\
2012-02-25

\paragraph{outline:}
Here are the points I want to make in this document:

1.~The four-second CCD transit time in \gaia\ has almost nothing to do
with the choice of how frequently there should be ``knots'' or control
points in the data-driven attitude model.  It is true that the mission
is not really sensitive to attitude issues on much smaller time scales
than four seconds, but there might be stability on mych longer time
scales.  This pre-launch choice of knot spacing precludes that kind of
(exceedingly valuable) discovery.

2.~Reducing model complexity, even though it reduces model freedom,
can often \emph{increase} the precision of the results.  The best
results are obtained with the optimal model complexity, where optimal
can be strictly defined in this kind of situation.

3.~The knot spacing---or any model-complexity control
parameter---should be set by an objective process involving either
optimization or marginalization.  It should not be set by intuition.

4.~There are three (easy) options for dealing with this issue:

4a.~Reduce large chunks of the data with different knot spacings (say
four seconds, eight, sixteen, and so on), using leave-one-out (or
really leave-subset-out) cross-validation to assess quality of the
attitude reconstruction in each case.  This is an low-assumption,
easy-to-justify, sensible, frequentist, engineering approach.

4b.~Stick with the current choice of four-second knot spacing, but add
into the least-squares solution a regularization (or prior) term that
prefers to put each knot at the mid-point (in attitude space) between
the adjacent knots.  Adjust the strength of this regularization to
optimize the leave-subset-out cross-validation likelihood or else some
kind of marginalized likelihood.  Or, in an extreme world, marginalize
over it (equivalent to marginalizing out a hyperparameter in a
hierarchical Bayesian model).

4c.~Go to an infinite-dimensional model that has (or really learns)
s/c solid-body physics and a (very large) set of angular impulse
parameters.  Never actually instantiate or report these parameters,
but work in an ``always marginalized'' framework in which only the
(physically motivated) PDF for the angular impulses is learned.  That
is, go hierarchical, learn the hyperparameters, and marginalize out
the infinite number of angular impulses.

Ideally, in this document, if I cared enough to send the very best, I
would:

5.~Build a toy one-dimensional model of the \gaia\ s/c, with a
one-dimensional sky (a ring of stars), active one-axis attitude
control, and an environment of random micro-impulses.

6.~Build a measurement model that reports noisy transit times for the
stars in the toy sky and s/c model.  Create many data sets under
different environment and control assumptions.

7.~Compete the \gaia\ default attitude model plan against the three
options labeled above as 4a, 4b, and 4c.  Show that all three either
match or outperform the default \gaia\ plan for all sets of
assumptions.  Make sure that this is true even when the modeling
assumptions don't match the simulating assumptions.

8.~Thank Berry Holl and Lennart Lindegren for getting me to think
about these things, and probably Coryn Bailer-Jones and Anthony Brown
for keeping me in the loop.

\end{document}
